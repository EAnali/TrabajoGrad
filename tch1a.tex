\chapter{Teoría Preliminar}

\section{Teoría de Probabilidad}

\subsection{Probabilidad}

\begin{defn}\textbf{(Experimento)}. Es el proceso por medio del cual se hace una observación.
	\end{defn}

\begin{defn}(Espacio Muestral). Sea \textit{S} un conjunto finito cuyos elementos son los posibles resultados de un experimento. A \textit{S} se le llamara \textit{espacio muestral}. A cada elemento de \textit{S} se llamarán \textit{puntos muestales}.
\end{defn}

\begin{defn}(Evento). Si \textit{S} es un espacio muestral, entonces un \textit{evento} es cualquier subconjunto de \textit{S}.	
\end{defn}
	
\begin{defn}(Unión). Si \textit{$A_{1}, A_{2}\subseteq S$} son eventos, entonces \textit{$A_{1}\cup A_{2}$} es la \textit{unión} de los conjuntos \textit{$A_{1}$} y \textit{$A_{2}$} que consiste de todos los puntos muestrales asociados a \textit{$A_{1}$} o \textit{$A_{2}$}. El evento \textit{$A_{1}\cup A_{2}$} ocurre si alguno de los eventos \textit{$A_{1}$} o \textit{$A_{2}$} ocurre.	
\end{defn}


\begin{defn}(Intersección). Si \textit{$A_{1}, A_{2} \subseteq S$} son eventos, entonces \textit{$A_{1}\cap A_{2}$} es la intersección de los conjuntos \textit{$A_{1}$} y \textit{$A_{2}$} que consiste de todos los puntos muestrales en \textit{$A_{1}$} y \textit{$A_{2}$}. El evento \textit{$A_{1}\cap A_{2}$} ocurre si los eventos \textit{$A_{1}$} y \textit{$A_{2}$} ocurren simultáneamente.	
\end{defn}

\begin{defn}(Mutuamente Excluyentes). Dos eventos \textit{$A_{1}, A_{2} \subseteq S$} se dice que son \textit{mutuamente excluyentes} si y solo si \textit{$A_{1}\cap A_{2} =\emptyset$}.
\end{defn}

\begin{defn}(Función de Distribución de Probabilidad (Discreta)). Dado un espacio muestral discreto \textit{S}, sea $\mathcal{\tilde{A}}$ el conjuntos de todos los eventos en \textit{S}. Una \textit{función de probabilidad discreta} es una aplicación \textit{P:$\mathcal{\tilde{A}} \longrightarrow [0, 1]$} con las siguientes propiedades:
	\begin{enumerate}
		\item P(\textit{S})=1
		\item Si \textit{$A, B\in \mathcal{\tilde{A}}$} y $A\cap B=\emptyset$, entonces \textit{P($A\cup B$)}=P($A$)+($B$).
	\end{enumerate}
\end{defn}

\begin{defn}(Espacio de Probabilida Discreto). La tripleta (\textit{S, $\mathcal{\tilde{A}}$, P}) se conoce como \textit{espacio de probabilidad} de \textit{S}.
	
\end{defn}

\begin{lem} \textit{Sea (S, $\mathcal{\tilde{A}}$, P) un espacio de probabilidad discreto. Entonces P($\emptyset$)= 0.}
\end{lem}

\begin{proof}
	Se tiene que $S\cap\emptyset=\emptyset$ entonces para P($S\cup \emptyset$ = S)= P(S)+P($\emptyset$), donde P(S)=1, entonces P($\emptyset$)=0.
\end{proof}


\begin{lem}Sea (S, $\mathcal{\tilde{A}}$, P) un espacio de probabilidad discreto y sean $A, B \in \mathcal{\tilde{A}}$. Entonces: \begin{equation}\label{punion}
		P(A\cup B)=P(A)+P(B)-P(A\cap B)
		\end{equation}
	
\end{lem}

\begin{proof}
	Si $A\cap B=\emptyset$ es trivial por definición (de función de distribución de probabilidad (discreta)) y lema anterior. Ahora supóngase que $A\cap B\neq\emptyset$, entonces sean:\\
	$E'={\omega\in E|\omega\notin F }$\\
	$F'={\omega\in F|\omega\notin E}$\\
	
	De donde se tiene lo siguiente:\\
	$E'\cap F'=\empty$\\
	$E'\cap(E\cap F)=\empty$\\
	$F'\cap(E\cap F)=\empty$\\
	$E=E'\cup(E\cap F)$ y\\
	$F=F'\cup(E\cap F)$\\
	
	Entonces por definicion de función de distribución de probabilidad (discreta) se tiene que:
	\begin{equation}\label{EyF}
	P(E\cup F)=P(E'\cup F'\cup (E\cap F))=P(E')+P(F')+P(E\cap F)
	\end{equation}\\
	\begin{equation}\label{E}
	P(E)=P(E'\cup (E\cap F))=P(E')+P(E\cap F)\Longrightarrow P(E')=P(E)-P(E\cap F)
	\end{equation}\\
	\begin{equation}\label{F}
	P(F)P(F'\cup (E\cap F))=P(F')+P(E\cap F)\Longrightarrow P(F')=P(F)-P(E\cap F)
	\end{equation}\\
	entonces al combinar las ecuaciones anteriores se tiene:\\
	\begin{equation}\label{key}
	P(E\cup F)=P(E)-P(E\cap F)+P(F)-P(E\cap F)+P(E\cap F)+P(E\cup F)=P(E)+P(F)-P(E\cap F)
	\end{equation}\\
	a donde se queria llegar.
\end{proof}


\begin{lem} Sea (S, $\mathcal{\tilde{A}}$, P)  un espacio de probabilidad discreto y sean $A, B\subseteq \mathcal{\tilde{A}}$. Entonces: \begin{equation}\label{pA}
	P(A)=P(A\cap B)+P(A\cap B^{c})
	\end{equation}
\end{lem} 

\begin{proof}
\end{proof}


\begin{lem} Sea (S, $\mathcal{\tilde{A}}$, P) un espacio de probabilidad y supongase que A,$B_{1},..., B_{n}$ son subconjuntos de S. Entonces:
	\begin{equation}\label{AcapcupBi}
	A\cap\bigcup\limits_{i=1}^nB_{i}	= \bigcup\limits_{i=1}^n(A\cap B_{i})
	\end{equation}
Es decir, la distribución de la intersección respecto de la unión.
\end{lem}

\begin{thm}Sea (S, $\mathcal{\tilde{A}}$, P) un espacio de probabilidad discreto y sean A $\in \mathcal{\tilde{A}}$. Sean $B_{1},..., B_{n}$ cualquier colección de conjuntos disjuntos entre si que particionan a \textit{S}. Es decir: \begin{equation}\label{cup Bi}
	S= \bigcup \limits_{i=1}^n Bi
	\end{equation}
 y $B_{i}\cap B_{j}=$ si $i\neq j$. Entonces:\begin{equation}\label{pA2} P(A)= \sum\limits_{i=1}^n(A\cap B_{i})
 \end{equation}
\end{thm}

\subsection{Variables Aleatorias y Valores Esperados}

\begin{defn} Sea (S, $\mathcal{\tilde{A}}$, P) un espacio de probabilidad discreto. Sea \textit{D$\subseteq\mathbb{R}$} un subconjunto discreto finito de números reales. Una variable aleatoria \textit{X} es una función que mapea cada elemento de \textit{S} en un elemento de \textit{D}. Formalmente \textit{$X:S\longrightarrow D$}.
\end{defn}

\begin{defn} Sea (S, $\mathcal{\tilde{A}}$, P) una distribución de probabilidad discreta y sea \textit{$X:S\longrightarrow D$} una variable aleatoria. Entonces el \textit{valor esperado} de \textit{X} es:
	\begin{equation}\label{esp}
	\mathbb{E}(X)=\sum\limits_{x\in D}xP(x)
	\end{equation}
\end{defn}

\subsection{Probabilidad Condicional}

\begin{lem} Sea (S, $\mathcal{\tilde{A}}$, P) un espacio de probabilidad discreto y supóngase un evento $A\subseteq S$. Entonces (A, $\mathcal{\tilde{A}}_{A}$, $P_{A}$) es un espacio de probabilidad discreta cuando:\begin{equation}\label{EspA} P_{A}(B)= \dfrac{P(B)}{P(A)}
	\end{equation}
para todo $B\subseteq A$ y $P_{A}(\omega)=0$ para todo $\omega\notin A$.

\end{lem}

\begin{defn}(Probabilidad Condicional) dado un espacio de probabilidad discreto (S, $\mathcal{\tilde{A}}$, P) y un evento $A\in\mathcal{\tilde{A}}$, la probabilidad condicional del evento  $B\in\mathcal{\tilde{A}}$ dado el evento A es:\begin{equation}\label{pcond}
	P(B\mid A)= \dfrac{P(B\cap A)}{P(A)}
	\end{equation}
\end{defn}

\begin{defn}(Independencia). Sea (S, $\mathcal{\tilde{A}}$, P) un espacio de probabilidad discreto. Dos eventos $A,B\in \mathcal{\tilde{A}}$ se dice que son independientes si $P(A\mid B)=P(A) y P(B\mid A)=P(B)$
\end{defn}

\begin{thm} Sea (S, $\mathcal{\tilde{A}}$, P) un espacio de probabilidad discreto. Si $A, B\in \mathcal{\tilde{A}}$ son eventos independientes, entonces $P(A\cap B)=P(A)P(B)$	
\end{thm}

\subsection{Regla de Bayes}

\begin{lem}(Teorema 1 de Bayes). Sea (S, $\mathcal{\tilde{A}}$, P) un espacio de probabilidad discreto y supóngase que $A, B\in\mathcal{\tilde{A}}$, entonces:\begin{equation}\label{bayes1} P(B\mid A)=\dfrac{P(A\mid B)P(B)}{P(A)}	
	\end{equation}
\end{lem}

\begin{thm}(Teorema 2 de Bayes). Sea (S, $\mathcal{\tilde{A}}$, P) un espacio de probabilidad discreto y supóngase que $A, B_{1},..., B_{n}\in\mathcal{\tilde{A}}$ con $B_{1},..., B_{n}$ una colección de conjuntos disjuntos y\\
	$S=\bigcup\limits_{i=1}^nB_{i}$\\
Entonces:\begin{equation}\label{bayes2}P(B_{i}\mid A)=\dfrac{P(A\mid B_{i})P(B_{i})}{\sum\limits_{j=1}^nP(A\mid B_{j})P(B_{j})}
\end{equation}
\end{thm}

\section{Teoría de la Utilidad}

\subsection{Funciones de Utilidad}

Se tiene un conjunto X, el cual consta de posibles alternativas, mutuamente excluyentes, entre las cuales debe de elegir un agente.

\begin{defn}(Preferencia). Sea $\succsim$ una relación binaria definida en X, llamada relación de preferencia, tal que, para x, y $\in$ X, x $\succsim$ y quiere decir que la alternativa x es preferida o indiferente ante la alternativa y.
	\end{defn}
	
\begin{defn}(Preferencia Estricta). La relación de preferencia estricta se denotada por el símbolo "$\succ$":
\begin{center}
		x$\succ$ y$\Leftrightarrow$ x $\succsim$ y, pero no y $\succsim$ x
\end{center}
	que se lee \textit{"x es preferido a y"}.
\end{defn}

\begin{defn}(Indiferencia). La relación de indiferencia se denota por el símbolo "$\sim$":\\
	$x \sim y \Leftrightarrow x\succsim y, y simultáneamente y\succsim x$\\
	que se lee \textit{"x es indiferente a y"}.
\end{defn}

\begin{defn}(Preferencia Racional). La relación de preferencia $\succsim$ es racional si se verifica lo siguiente:
	\begin{itemize}
		\item Completitud: dadas dos alternativas cualesquiera x, y  son comparables entre sí, en el sentido que es preferida x, es preferida y o son indiferentes.\\
		 $\forall x, y \in X, se tiene que x\succsim y o y\succsim x o x \sim y $
		\item Transitividad: dadas alternativas cualesquiera x, y, z si x es preferida que y e y es preferiza que z, entonces x se prefiere que z.\\
		 $\forall x, y, z\in X, si x\succsim y e y\succsim z, entonces x\succsim z$.
	\end{itemize}
\end{defn}

\begin{defn}(Función de Utilidad). La función U que asigna números a alternativas es la función de utilidad del agente sobre X. $U: X\longrightarrow R$ es una función de utilidad que representa la relación de preferencia $\succsim$, si para todo $x, y\in X, x\succsim y \Leftrightarrow U(x)\geq U(y)$.
\end{defn}

\subsection{Toma de Decisiones con Seguridad }

Sea {$X={x_{1}, x_{2}, ..., x_{n}}$ un conjunto finito de alternativas en un ambiente de riesgo.

\begin{defn}(Lotería). L es una lotería simple en X, si:\\
	$L=<{x_{1}, x_{2}, ..., x_{n}}, P>$\\
es decir, el conjunto de alternativas 
\end{defn}





\section{Teoría de Juegos}

\subsection{Conceptos Básicos}

\begin{defn}(Juego). Sean X, Y dos jugadores con intereses opuestos. Un juego es definido como el curso de eventos que consisten de una sucesión de acciones por parte de X, Y. Para que dicho juego sea suceptible de análisis matemático, debe tener un sistema de reglas bien definidas, es decir, un sistema de condiciones que establezcan las acciones permisibles para cada jugador en cada etapa del juego.\\
\end{defn}	

\begin{figure}[h]
	\center
	\label{fig.juegodeajedrez}
		\newgame   % Comenzamos una nueva partida
	\showboard % Y mostramos el tablero a continuación
	\caption[Juego del Ajedrez]{El juego del ajedrez es objeto de análisis matemático en la Teoría de Juegos.}
\end{figure}


\begin{defn}(Jugadores). Son los participantes en el juego que toman decisiones con el fin de maximizar su utilidad. Pueden ser dos o más.
\end{defn}

\begin{defn}(Jugadas). Son las decisiones que puede tomar cada jugador en cada momento en que le toquen jugar. El conjunto de jugadas en cada momento del juego puede ser finito o infinito.
\end{defn}

\begin{defn}(Jugada Personal). Es una elección y ejecución consciente, por parte de uno de los jugadores, de una de las jugadas que sean posibles en la situación dada.
\end{defn}

\begin{defn}(Jugada Aleatoria). Es la elección de una posibilidad de entre un cierto número de ellas, no por la decisión de un jugador, sino por el resultado de algún evento aleatorio.
\end{defn}

\begin{defn}(Resultados del Juego). Son los distintos modos en que puede concluir un juego. Cada resultado lleva aparejadas unas consecuencias para cada jugador.
\end{defn}

\begin{defn}(Pago). Es la valoración que obtiene un jugador al final del juego, la cual está asociada a las consecuencias de alcanzar un determinado resultado.
\end{defn}

\begin{defn}(Estrategia). Es el conjunto completo de las reglas que determinan sus elecciones para todas las situaciones que se presentan en el curso de un juego.
\end{defn}

\begin{defn}(Perfiles de Estrategias). Es un conjunto de estrategias, uno por cada jugador.
\end{defn}

