\documentclass{article}
\usepackage[utf8]{inputenc}
\usepackage{lmodern}
\usepackage[T1]{fontenc}
\usepackage[spanish,activeacute]{babel}
\usepackage{mathtools}
\usepackage{amsfonts,amsmath,amssymb,amsthm,mathrsfs}

\begin{document}

\begin{flushleft}
	Universidad de San Carlos de Guatemala\\
	Escuela de Ciencias Físicas y Matemáticas\\
	Análisis Funcional 2\\
	Emilene Analí Romero\\
	Carné: 201213113
\end{flushleft}

\begin{center}
	\textit{\underline{Función Generatriz de los Polinomios de Legendre}}
\end{center}
	
\textbf{(Función Generatriz)} Mostrar que:

\begin{equation*} %esta bien
\dfrac{1}{\sqrt{1-2wt+w^{2}}}=\sum\limits_{n=0}^\infty P_{n}(t)w^{n}
\end{equation*}

La función de la izquierda es conocida como \textit{la función generatriz} de los polinomios de Legendre.\\
	
\textbf{Solución:}\\
	
Se tiene que la función $f(w,t)=(1-2wt+w^{2})^{-1/2}$, puede escribirse como:
	
\begin{equation*} %esta bien
(1-2wt+w^{2})^{-1/2}=[1-w(2t-w)]^{-1/2}
\end{equation*}
	
Entonces haciendo uso de la siguiente fórmula: 
	
\begin{equation*} %esta bien
\sum\limits_{n=0}^\infty\dfrac{(\alpha)_{n}}{n!}z^{n}=(1-z)^{-\alpha}
\end{equation*}

entonces al reemplazar $z=2wt+w^{2}$ y $\alpha=\dfrac{1}{2}$ se obtiene:

\begin{equation*}
(1-2wt+w^{2})^{-1/2}=\sum\limits_{n=0}^\infty\dfrac{(1/2)_{n}}{n!}(2wt+w^{2})^{n}=\sum\limits_{n=0}^\infty\dfrac{(1/2)_{n}}{n!}w^{n}(2t+w)^{n}
\end{equation*}	

de donde se tiene que:\\

\begin{equation*}
\begin{array}{lll}
(1/2)_{n}=\dfrac{\Gamma(n+1/2)}{\Gamma(1/2)}=\dfrac{1\cdot3\cdot5\cdot\cdots\cdot(2n-1)}{2^{2}} &para &n=1,2,...
\end{array}
\end{equation*}

por lo que:\\

\begin{equation*}
\begin{array}{lll}
(1-2wt+w^{2})^{-1/2}&=&1+\sum\limits_{n=0}^\infty\dfrac{1\cdot3\cdot5\cdots\cdot(2n-1)}{2^{n}n!}w^{n}(2t-w)^{n}\\
 &=&1+\dfrac{1}{2\cdot1!}w(2t-w)+\dfrac{1\cdot3}{2^{2}2!}w^{2}(2t-w)^{2}+\dfrac{1\cdot3\cdot5}{2^{3}3!}w^{3}(2t-w)^{3}+\\
  & &+...+\dfrac{1\cdot3\cdot5\cdots(2n-3)}{2^{n-1}(n-1)!}w^{n-1}(2t-w)^{n-1}+\\
   & &+\dfrac{1\cdot3\cdot5\cdots(2n-1)}{2^{n}n!}w^{n}(2t-w)^{n}+... 
\end{array}
\end{equation*}

Sean $a_{n}(t)$ los coeficientes de $t^{n}$ entonces se tiene que:\\

\begin{equation*}
	\begin{array}{lll}
	a_{n}(t)&=&\dfrac{1\cdot3\cdot5\cdots(2n-1)}{2^{2}n!}(2t)^{n}-\dfrac{1\cdot3\cdot5\cdots(2n-3)}{2^{n-1}(n-1)!}\dfrac{(n-1)}{1!}(2t)^{n-2}+\\
	& &+\dfrac{1\cdot3\cdot5\cdots(2n-5)}{2^{n-2}(n-2)!}\dfrac{(n-2)(n-3)}{2!}(2t)^{n-4}-+\cdots
	\end{array}
\end{equation*}

Entonces haciendo uso de la siguiente identidad:\\

\begin{equation*}
1\cdot3\cdot5\cdots(2n-1)=\dfrac{(2n)!}{2^{n}n!}
\end{equation*}

de donde $a_{n}(t)$ se reduce a:\\

\begin{equation*}
\begin{array}{lll}
a_{n}(t)&=&\dfrac{(2n)!}{2^{n}n!n!}t^{n}-\dfrac{(2n-2)!}{2^{n}1!(n-1)!(n-2)!}t^{n-2}\\
&&+\dfrac{(2n-4)!}{2^{n}2!(n-2)!(n-4)!}t^{n-4}-+\cdots\\
&=&\sum\limits_{j=0}^{n}(-1)^{j}\dfrac{(2n-2j)!}{2^{n}j!(n-j)!(n-2j)!}t^{n-2j}\\
&=&P_{n}(t)
\end{array}
\end{equation*}



\end{document}