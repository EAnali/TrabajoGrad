\documentclass[10pt]{beamer}
\usepackage[spanish,es-tabla]{babel}
\usepackage[utf8]{inputenc}

\usetheme[progressbar=frametitle]{metropolis}

\usepackage{booktabs}
\usepackage[scale=2]{ccicons}
\usepackage[spanish,es-tabla]{babel}

\usepackage{pgfplots}
\usepgfplotslibrary{dateplot}

\usepackage{xspace}
\newcommand{\themename}{\textbf{\textsc{metropolis}}\xspace}

\title{Punto de Tesis}
\subtitle{}
\date{Guatemala, \today}
\author{Emilene A. Romero M.}
\institute{Universidad de San Carlos de Guatemala}
\titlegraphic{\hfill\includegraphics[height=1.5cm]{escudo}}

\begin{document}

\maketitle

\begin{frame}{Tabla de Contenidos}
  \setbeamertemplate{section in toc}[sections numbered]
  \tableofcontents[hideallsubsections]
\end{frame}

\section{Posibles Puntos de Tesis}

\begin{frame}{Opciones}
\begin{itemize}
	\item[1.] La valuación de los planes de pensiones ocupacionales o privados.
	\item[2.] La seguridad social pública y privada
\end{itemize}
\end{frame}

\subsection{1. La valuación de los planes de pensiones ocupacionales o privados.}
\begin{frame}{1. La valuación de los planes de pensiones\\
		 ocupacionales o privados.}

	\begin{itemize}
		\item [1. 1.] Formas para establecer la situación financiera de un
		plan de pensiones. 
		\begin{itemize}
			\item [1. 1. a] El flujo de efectivo (ingreso menos gastos) positivo en un período de tiempo (equilibrio). 
			\item [1. 1. b] El financiamiento de su Reserva Matemática.
		\end{itemize}
		\item [1. 2.]\textbf{Objetivo}:
		\begin{itemize}
			\item Revisar los aspectos matemáticos de ambos métodos, y comprobar la conveniencia de su aplicabilidad en los planes de pensiones ocupacionales.
		\end{itemize}
	\end{itemize}
\end{frame}
\subsection{2. La seguridad social pública y privada}
\begin{frame}{2. La seguridad social pública y privada}
	\begin{itemize}
	\item[2. 1.] Todos los países tienen su seguridad social compuesta por dos partes, una pública y otra privada con las empresas aseguradoras.
	\begin{itemize}
		\item[2. 1. a.] Algunos economistas como los neo-liberales dicen que esto es ineficiente y que va contra el libre mercado, y sugieren que toda la	seguridad social sea prestada por las empresas privadas.
		\item[2. 1. b.] Sin embargo otros economistas opinan lo contrario y abogan por el fortalecimiento de los sistemas púbicos de seguridad social.
	\end{itemize}
		\item[2. 2.] \textbf{Objetivo}:
		\begin{itemize}
			\item Determinar mediante el análisis de la teoría de juegos cuál de los dos escenarios es mejor.
		\end{itemize}
	\end{itemize}
\end{frame}

%\section{Bibliografía Preliminar}

%\begin{frame}{Bibliografía Preliminar}
%	\bibliography{bibliotesis}
%	\end{frame}

\end{document}