\documentclass[10pt]{beamer}
\usepackage[spanish,es-tabla]{babel}
\usepackage[utf8]{inputenc}

\usetheme[progressbar=frametitle]{metropolis}

\usepackage{booktabs}
\usepackage[scale=2]{ccicons}
\usepackage[spanish,es-tabla]{babel}

\usepackage{pgfplots}
\usepgfplotslibrary{dateplot}

\usepackage{xspace}
\newcommand{\themename}{\textbf{\textsc{metropolis}}\xspace}

\title{Teoría de Juegos}
\subtitle{}
\date{Guatemala, \today}
\author{Emilene A. Romero M.}
\institute{Universidad de San Carlos de Guatemala}
\titlegraphic{\hfill\includegraphics[height=1.5cm]{escudo}}

\begin{document}

\maketitle

\begin{frame}{Tabla de Contenidos}
  \setbeamertemplate{section in toc}[sections numbered]
  \tableofcontents[hideallsubsections]
\end{frame}

\section{Historia de la Teoría de Juegos}

\begin{frame}[fragile]{Principales Aportaciones}
	\begin{itemize}
		\item John Von Neumann y Oskar Morgenstern (1944) \textit{The Theory of Games and Economic Behavior}
		\item James Waldegrave (1713) \textit{Teorema del Minimax}
		\item Ernst Zermelo (1913) \textit{El ajedrez está estrictamente determinado}
		\item Emile Borel (1921-1927) \textit{"Solución Minimax", dos personas y 3-5 estrategias posbiles}
		\item John Nash (1953) \textit{"Teoía de Juegos No cooperativos y a la teoría de negociación"}
		
	\end{itemize}
\end{frame}

\section{Teoría de Probabilidad}

\section{Teoría de la Utilidad}

\section{Conceptos básicos de la Teoría de Juegos}

\begin{frame}[fragile]{Definiciones Previas}
	
	Sean A y B jugadores, con intereses opuestos.

 \begin{itemize}
 	\item \textit{Juego}
 	\item\textit{Tipos de Juego}
 	\item\textit{Jugada}
 	\item\textit{Tipos de Jugada}
 	
 	\end{itemize}
 	
  
\end{frame}
\begin{frame}[fragile]{Definiciones Previas}
  
  \begin{itemize}
  \item\textit{Estrategia}
  \item \textit{Resultado}
  \item \textit{Juego de suma cero}
  \item\textit{Juego de información perfecta}


  	\end{itemize}
  
\end{frame}

\begin{frame}[fragile]{Definiciones Previas}
	
	\begin{itemize}
		\item\textit{Juegos de mxn}\\
		\begin{table}
			\begin{tabular}{|c|c|c|c|c|}
				\hline 
				$A \diagdown B$ & $B_{1}$ & $B_{2}$ & ... & $B_{n}$ \\ 
				\hline 
				$A_{1}$ & $a_{11}$ & $a_{12}$ & ... & $a_{1n}$ \\ 
				\hline 
				$A_{2}$ & $a_{21}$ & $a_{22}$ & ... & $a_{2n}$ \\ 
				\hline 
				$\colon$& $\colon$ & $\colon$ & $\ddots$ & $\colon$ \\ 
				\hline 
				$A_{m}$ & $a_{m1}$ & $a_{m2}$ & ... & $a_{mn}$ \\ 
				\hline 
			\end{tabular} 
		\end{table}
		\item\textit{Estrategia Óptima}
		\item\textit{Estrategia Mixta}
		
	\end{itemize}
	
\end{frame}
\begin{frame}[fragile]{Definiciones Previas}
	
	\begin{itemize}
		\item\textit{El Principio \textbf{\textit{minimax}}}: 
		\begin{table}
			\begin{tabular}{|c|c|c|c|c||c|}
				\hline 
				$A \diagdown B$ & $B_{1}$ & $B_{2}$ & ... & $B_{n}$ & $\alpha_{i}$ \\ 
				\hline 
				$A_{1}$ & $a_{11}$ & $a_{12}$ & ... & $a_{1n}$ & $\alpha_{1}$ \\ 
				\hline 
				$A_{2}$ & $a_{21}$ & $a_{22}$ & ... & $a_{2n}$ & $\alpha_{2}$ \\ 
				\hline 
				... & ... & ... & ... & ... & ... \\ 
				\hline 
				$A_{m}$ &$ a_{m1}$ & $a_{m2}$ & ... & $a_{mn}$ & $\alpha_{m}$ \\ 
				\hline 
				\hline
				$\beta_{j}$ & $\beta_{1}$ & $\beta_{2}$ & ... & $\beta_{n}$ &\\
				\hline
			\end{tabular} 
		\end{table}
		\item \textit{maximin} $\alpha=\smash{\displaystyle\max_{j}\min_{i}}$ $a_{ij}$
		\item \textit{minimax} $\beta=\smash{\displaystyle\min_{j}\max_{i}}$ $a_{ij}$
		\item \textit{punto silla} $\alpha=\beta$
		
		
	\end{itemize}
	
	
\end{frame}

\begin{frame}[fragile]{Definiciones Previas}	
	\begin{itemize}
		\item\textit{Ejemplo 1:} Cada uno de los jugadores A y B escribe un número, 1,2 ó 3, simultáneamente e independientemente. Si la suma de los dos números es par, B paga a A la suma en quetzales; si la suma de los números es impar, A paga a B.
		\begin{table}
			\begin{tabular}{|c|c|c|c||c|}
				\hline 
				$A \diagdown B$ & $B_{1}$ & $B_{2}$ & $B_{3}$ & $\alpha_{i}$ \\ 
				\hline 
				$A_{1}$ & 2 & -3 & 4 & -3 \\ 
				\hline 
				$A_{2}$ & -3 & 4 & -5 & -5  \\ 
				\hline 
				$A_{3}$& 4 & -5 & 6 & -5 \\ 
				\hline 
				\hline
				$\beta_{j}$ & 4 & 4 & 6 &  \\
				\hline
			\end{tabular} 
		\end{table}	
	\end{itemize}
\end{frame}
\begin{frame}[fragile]{Definiciones Previas}	
	\begin{itemize}
		\item\textit{Ejemplo 2:} 
		\begin{table}
			\begin{tabular}{|c|c|c|c|c||c|}
				\hline 
				$A \diagdown B$ & $B_{1}$ & $B_{2}$ & $B_{3}$ & $B_{4}$ & $\alpha_{i}$ \\ 
				\hline 
				$A_{1}$ & 0.4 & 0.5 & 0.9 & 0.3 & 0.3 \\ 
				\hline 
				$A_{2}$ & 0.8 & 0.4 & 0.3  & 0.7 &  0.3 \\ 
				\hline 
				$A_{3}$& 0.7 & 0.6 & 0.8 &  0.9 &  \textbf{0.6} \\ 
				\hline 
				$A_{4}$& 0.7 & 0.2 & 0.4 & 0.6 & 0.2\\
				\hline
				$\beta_{j}$ & 0.8 & \textbf{0.6} & 0.9 & 0.9 &  \\
				\hline
			\end{tabular} 
		\end{table}	
	\end{itemize}
\end{frame}
\end{document}