\chapter{PLANES DE PENSIONES }

En Guatemala los planes de pensiones ocupacionales, son pensiones provisionales otorgados por las empresas a sus trabajadores como complementos de las brindadas por el Instituto Guatemalteco de Seguridad Social (IGSS). 


\section{ASPECTO HISTÓRICO}

Durante el siglo XIX con el surgimiento de la Revolución Industrial, en los países conocidos como desarrollados, en Europa o Estados Unidos por ejemplo,  se promovió la creación de una institución que velara por la seguridad económica del trabajador, en caso de este encontrarse en una situación desventajosa para poder proveerse de un modo para obtener ingresos, siendo esta la \textbf{seguridad social}. 

Entre los beneficios que la seguridad social se encuentra lo que son los \textbf{planes de pensiones}, los cuales son convenios institucionales que protegen en la vejez, la invalidez y en el caso de fallecimiento, del proveedor de sustento del hogar, a los dependientes quienes sufren la pérdida. Con el tiempo distintintas instituiones privadas fueron implementando un programa similar para sus trabajadores o para un grupo en específico, como lo que proveen ciertos colegios de profesionales. Los planes de pensiones se denominan \textbf{sociales} cuando estos son administrados por el seguro social y \textbf{ocupacionales} cuando su administración corresponde a una entidad privada.  En Guatemala, los planes de pensiones ocupacionales son otorgados por algunas empresas a sus trabajadores, como complemento de los beneficios brindados por el Insituto Guatemalteco de Seguridad Social (IGSS). 

\section{BENEFICIOS DEL CONTRIBUYENTE}

\subsection{Beneficio de Vejez}
\subsection{Beneficio de Invalidez}

Un asegurado se cataloga en beneficio de \textit{Invelidez:} si: presenta pérdida parcial o total de sus capacidades, sea esta física o mental; denominando a este estado \textbf{invalidez física}, si presenta incapacidad de proveerse de un ingreso 

\subsection{Benificio de Supervivencia}


\section{REGÍMENES DE LOS PLANES DE PENSIONES}



